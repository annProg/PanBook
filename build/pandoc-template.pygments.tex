\documentclass[fancyhdr,bookmark]{ctexbook}
\setCJKmainfont{SimSun}
\setmainfont{Georgia} 	% 設定英文字型
\setromanfont{Georgia} 	% 字型
\setmonofont{Latin Modern Mono}

\usepackage{tikz} % Required for drawing custom shapes
\usepackage[yyyymmdd,hhmmss]{datetime}
\ctexset{today=small}
\usepackage{geometry} 		% 設定邊界
\geometry{
  top=1in,
  inner=1in,
  outer=1in,
  bottom=1in,
  headheight=3ex,
  headsep=2ex
}


\usepackage{xcolor}
\definecolor{ocre}{RGB}{243,102,25} % Define the orange color used for highlighting throughout the book

%\usepackage[x11names,svgnames,dvipsnames]{xcolor}
\usepackage{listings}
\lstset{
	%numbers=left,
	%numberstyle=\tiny,
	basicstyle=\small\ttfamily,
	keywordstyle=\color[rgb]{0.13,0.29,0.53}\textbf,
	commentstyle=\color[rgb]{0.56,0.35,0.01}\textit,
	identifierstyle=\color[rgb]{0.00,0.00,0.00},
	stringstyle=\color[rgb]{0.31,0.60,0.02},
	frame=shadowbox,
	rulesepcolor=\color{red!20!green!20!blue!20},
	backgroundcolor=\color[rgb]{0.97,0.97,0.97},
	tabsize=4,
	breaklines=tr,
	showstringspaces=false,
}
\renewcommand{\lstlistingname}{代码}

%\newcommand{\KeywordTok}[1]{\textcolor[rgb]{0.13,0.29,0.53}{\textbf{{#1}}}}
%\newcommand{\DataTypeTok}[1]{\textcolor[rgb]{0.13,0.29,0.53}{{#1}}}
%\newcommand{\DecValTok}[1]{\textcolor[rgb]{0.00,0.00,0.81}{{#1}}}
%\newcommand{\BaseNTok}[1]{\textcolor[rgb]{0.00,0.00,0.81}{{#1}}}
%\newcommand{\FloatTok}[1]{\textcolor[rgb]{0.00,0.00,0.81}{{#1}}}
%\newcommand{\CharTok}[1]{\textcolor[rgb]{0.31,0.60,0.02}{{#1}}}
%\newcommand{\StringTok}[1]{\textcolor[rgb]{0.31,0.60,0.02}{{#1}}}
%\newcommand{\CommentTok}[1]{\textcolor[rgb]{0.56,0.35,0.01}{\textit{{#1}}}}
%\newcommand{\OtherTok}[1]{\textcolor[rgb]{0.56,0.35,0.01}{{#1}}}
%\newcommand{\AlertTok}[1]{\textcolor[rgb]{0.94,0.16,0.16}{{#1}}}
%\newcommand{\FunctionTok}[1]{\textcolor[rgb]{0.00,0.00,0.00}{{#1}}}
%\newcommand{\RegionMarkerTok}[1]{{#1}}
%\newcommand{\ErrorTok}[1]{\textbf{{#1}}}
%\newcommand{\NormalTok}[1]{{#1}}



\ifxetex
  \usepackage[setpagesize=false, % page size defined by xetex
              unicode=false, % unicode breaks when used with xetex
              xetex]{hyperref}
\else
  \usepackage[unicode=true]{hyperref}
\fi
\hypersetup{breaklinks=true,
            bookmarks=true,
            pdfauthor={An He},
            pdftitle={用Markdown+Pandoc+XeLaTeX写作},
            colorlinks=true,
            urlcolor=blue,
            linkcolor=magenta,
            pdfborder={0 0 0}}
\urlstyle{same}  % don't use monospace font for urls
% Make links footnotes instead of hotlinks:
\renewcommand{\href}[2]{#2\footnote{\url{#1}}}

\usepackage{longtable,booktabs}


\title{用Markdown+Pandoc+XeLaTeX写作}
\author{An He}
\date{\today}
\usepackage{cleveref}

\usepackage{fancyhdr}
%\usepackage{lastpage}
\pagestyle{fancy}


\begin{document}
\frontmatter
%----------------------------------------------------------------------------------------
%	TITLE PAGE
%----------------------------------------------------------------------------------------

\begingroup
\thispagestyle{empty}
\begin{tikzpicture}[remember picture,overlay]
\node[inner sep=0pt] (background) at (current page.center) {\includegraphics[width=\paperwidth]{Pictures/background}};
\draw (current page.center) node [fill=ocre!30!white,fill opacity=0.6,text opacity=1,inner sep=1cm]
{\Huge\centering\bfseries\sffamily\parbox[c][][t]{\paperwidth}
{\centering 用Markdown+Pandoc+XeLaTeX写作\\[13pt] % Book title
{\huge An He} % Author name
}};
\end{tikzpicture}
\vfill
\endgroup
\addcontentsline{toc}{chapter}{封面}

%----------------------------------------------------------------------------------------
%	COPYRIGHT PAGE
%----------------------------------------------------------------------------------------
\newpage
~\vfill
\thispagestyle{empty}

\noindent Copyright \copyright\ \the\year\  An He\\ % Copyright notice

\noindent \textsc{Published by \LaTeX}\\ % Publisher
\noindent \textsc{https://github.com/annProg/pandoc-template}\\ % URL

\noindent Licensed under the Creative Commons Attribution-NonCommercial 3.0 Unported License (the ``License''). You may not use this file except in compliance with the License. You may obtain a copy of the License at \url{http://creativecommons.org/licenses/by-nc/3.0}. Unless required by applicable law or agreed to in writing, software distributed under the License is distributed on an \textsc{``as is'' basis, without warranties or conditions of any kind}, either express or implied. See the License for the specific language governing permissions and limitations under the License.\\ % License information

\noindent \textit{最后编译日期, \today\ \currenttime } % Printing/edition date


    
\chapter*{前言}
\addcontentsline{toc}{chapter}{前言}
\LaTeX~可以排版格式精美的书籍,但是学习成本较高,使用不便;
Markdown是一种简单易学的标记语言。如果能结合两者的优点,使用Markdown
来写作,然后通过程序转换为latex源码,再编译为PDF,那将是一件美妙的事情。
幸运的是,已经有工具很好的实现了支持这一功能,那就是Pandoc。

Pandoc是由 John MacFarlane 教授开发的标记语言转换工具,实现了数十种标
记语言之间的转换。Pandoc还扩展了Markdown语法,比如标题表格等的ID属性,
脚注等,并且可以直接嵌入LaTeX代码,这样在Markdown中就可以实现输入数学
公式,交叉引用等功能。Pandoc还支持自定义转换模板,通过命令\lstinline!pandoc -D latex!可以输出默认的LaTeX模板,以此模板为基础,可以定制自己的模板。

本工具定义了一种Markdown源码组织规范,提供了一个转换脚本,用来更方便的
使用Pandoc将Markdown转换为PDF。另外还定义了一套LaTeX书籍模板,用来生成
中文书籍。用户也可以在自己的工作目录修改此模板,并通过修改配置来引用自己
的模板。


{
\hypersetup{linkcolor=black}
\setcounter{tocdepth}{2}
\tableofcontents
\addcontentsline{toc}{chapter}{目录}
}
\listoftables
\addcontentsline{toc}{chapter}{表格列表}
\listoffigures
\addcontentsline{toc}{chapter}{插图列表}



\mainmatter
    % 在此命令之后的页码为阿拉伯数字
    % 以下为正文
\chapter{使用说明}\label{ux4f7fux7528ux8bf4ux660e}

\section{安装步骤}\label{ux5b89ux88c5ux6b65ux9aa4}

首先克隆代码库

\begin{lstlisting}
git clone https://github.com/annProg/pandoc-template
\end{lstlisting}

然后将pandoc-template目录加入环境变量

建立工作目录

\begin{lstlisting}
mkdir workdir
cd workdir
bookgen.sh init # 初始化工作环境
\end{lstlisting}

目录结构说明

\begin{lstlisting}
.
├── book                                    # 书籍模板,暂未用到
│   ├── book-template.latex
│   └── pm-template.latex
├── bookgen.sh                              # 转换脚本
├── build                                   # 电子书构建目录
├── config.default                          # pandoc默认转换配置
├── html5                                   # html5电子书模板
├── README.md
├── resume                                  # 简历模板
├── src                                     # Markdown源码目录
│   └── images                              # 源码涉及插图目录
└── zh-ctex                                 # ctexbook模板目录
    ├── Pictures                            # 模板引用的图片资源
\end{lstlisting}

\section{使用规范}\label{ux4f7fux7528ux89c4ux8303}

\subsection{源码命名规范}\label{ux6e90ux7801ux547dux540dux89c4ux8303}

脚本中使用\lstinline!ls src/*.md!列出所有的Markdown源码,要保证顺序正确,才能生成正确的LaTeX源码。
因此,要求Markdown源码文件命名能够被ls以正确的顺序列出。例如,有少于十个的Markdown文件,
可以使用0\textasciitilde{}9为前缀:

\begin{lstlisting}
$ tree src/
src/
├── 0-title.md
├── 1-intro.md
├── 2-pandoc-markdown.md
├── 3-template.md
└── images
\end{lstlisting}

如果Markdown文件数多于10个,则需要在前缀为个位数的前面补0,与最大前缀数字位数保持一致,例如
最后一个Markdown文件为\lstinline!99-markdown.md!,那么个位数应形如
\lstinline!01-first.md!。

\subsection{编码规范}\label{ux7f16ux7801ux89c4ux8303}

Markdown源码文件需要使用UTF-8编码。以Notepad++为例,依次选择\textbf{格式,以UTF-8无BOM格式编码}
即可正确设置编码。

\subsection{注意事项}\label{title:note}

Pandoc扩展的Markdown语法要求在标题前留出一个空行,因此按章节拆分的多个Markdown文件,开头需要
空一行,否则pandoc不能正确识别标题。

\section{模板变量说明}\label{ux6a21ux677fux53d8ux91cfux8bf4ux660e}

可以使用\href{http://www.ruanyifeng.com/blog/2016/07/yaml.html}{Yaml语言}
定义模板中的变量,建议第一个
Markdown文件专门用来定义变量,如代码\ref{code:template-var}所示。

\begin{lstlisting}[label=code:template-var, caption=code:template-var, float=htbp]
---
title: 用Markdown+Pandoc+XeLaTeX写作
author:          # 作者(数组)
  - An He
date: \today     # 日期
copyright: true  # 是否生成版权页
lof: true        # 是否生成插图列表页
lot: true        # 是否生成表格列表页
homepage: https://github.com/annProg/pandoc-template
header-includes:
  - \usepackage{cleveref}
# preface用于生成前言
preface: '\LaTeX\ 可以排版格式精美的书籍,但是学习成本较高,使用不便;
 换行请在开头留出一个空格'
---
\end{lstlisting}

其中title,author,date 变量也可以通过以下形式来定义:

\begin{lstlisting}
% title
% author(s) (separated by semicolons)
% date
\end{lstlisting}

查看模板文件,可以获取所有变量(形如\lstinline!$var$!)。也可以通过修改模板来添加自定义的变量。

\section{转换命令}\label{ux8f6cux6362ux547dux4ee4}

pandoc-template目录加入环境变量后可以直接调用\lstinline!bookgen.sh!:

\begin{lstlisting}
bookgen.sh init  # 初始化工作环境
bookgen.sh pdf   # 生成pdf电子书
bookgen.sh html  # 生成html电子书
bookgen.sh pdf d # 调试模式,只使用一个代码高亮风格, html电子书也支持调试模式
\end{lstlisting}

\chapter{Pandoc
Markdown语法简介}\label{pandoc-markdownux8bedux6cd5ux7b80ux4ecb}

Pandoc 的目标与原始Markdown 的最初目标有着方向性的不同。在Markdown
原本的设计中, HTML 是其主要输出对象;然而Pandoc
则是针对多种输出格式而设计。因此,虽然Pandoc 同样也允许直接嵌入HTML
标签,但并不鼓励这样的作法,取而代之的是Pandoc 提供了许多 非HTML
的方式,来让使用者输入像是定义列表、表格、数学公式以及脚注等诸如此类的重
要文件元素。

Pandoc
Markdown语法介绍可以在\href{http://www.pandoc.org/MANUAL.html\#pandocs-markdown}{Pandoc主页}
找到。以下翻译大部分部分摘自\href{http://pages.tzengyuxio.me/pandoc/}{tzengyuxiao的翻译},
在此向译者表示感谢。

\section{段落}\label{ux6bb5ux843d}

一个段落指的是一行以上的文字,跟在一行以上的空白行之后。换行字元会被当作是空白处
理,因此你可以依自己喜好排列段落文字。如果你需要强制换行,在行尾放上两个以上的空
白字元即可。

\subsubsection{Extension:
escaped\_line\_breaks}\label{extension-escapedux5flineux5fbreaks}

一个反斜线后跟着一个换行字元,同样也有强制换行的效果。这也是在表格单元格中添加换
行的唯一形式。

\section{标题}\label{ux6807ux9898}

有两种不同形式的标题语法,Setext
以及Atx。Setext风格的标题是由一行文字底下接着一
行=符号(用于一阶标题)或-符号(用于二阶标题)所构成;Atx风格的标题是由一到六个\#符
号以及一行文字所组成,你可以在文字后面加上任意数量的\#符号。由行首起算的\#符号数量决
定了标题的阶层,如代码\ref{code:markdownTitle}所示。

\begin{lstlisting}[label=code:markdownTitle, caption=code:markdownTitle, float=htbp]
Setext A level-one header
==================

Setext A level-two header
------------------

# Atx level-one

## Atx level-two

### Atx  level-three
\end{lstlisting}

\subsubsection{Extension:
blank\_before\_header}\label{extension-blankux5fbeforeux5fheader}

原始markdown语法在标题之前并不需要预留空白行。Pandoc则需要(除非标题位于文件最开始的
地方)。这是因为以\#符号开头的情况在一般文字段落中相当常见,这会导致非预期的标题。例如:

\begin{lstlisting}
I like several of their flavors of ice cream:
#22, for example, and #5.
\end{lstlisting}

这也是前一章所述注意事项\ref{title:note}的原因。

\section{HTML与LaTeX的标题标识符}\label{htmlux4e0elatexux7684ux6807ux9898ux6807ux8bc6ux7b26}

\subsubsection{Extension:
header\_attributes}\label{extension-headerux5fattributes}

在标题文字所在行的行尾,可以使用以下语法为标题加上属性:

\begin{lstlisting}
{#identifier .class .class key=value key=value}
\end{lstlisting}

虽然这个语法也包含加入类别(class)以及键/值形式的属性(attribute),
但目前只有标识符(identifier/ID)在输出时有实际作用(且只在部分格式
的输出,包括:HTML, LaTeX, ConTeXt, Textile, AsciiDoc)。举例来说,
下面是将标题加上foo标识符的几种方法:

\begin{lstlisting}
# My header {#foo}

## My header ##    {#foo}

My other header   {#foo}
---------------
\end{lstlisting}

(此语法与PHP Markdown Extra相容。)

具有unnumbered类别的标题将不会被编号,即使--number-sections的选项是开启
的。单一连字符号( -)等同于.unnumbered,且更适用于非英文文件中。因此,

\begin{lstlisting}
# My header {-}
\end{lstlisting}

与下面这行是等价的

\begin{lstlisting}
# My header {.unnumbered}
\end{lstlisting}

\section{引用}\label{ux5f15ux7528}

Markdown使用email的习惯来建立引用区块。一个引用区块可以由一或多个段落
或其他的区块元素(如列表或标题)组成,并且其行首均是由一个\textgreater{}符号加上一
个空白作为开头。(\textgreater{}符号不一定要位在该行最左边,但也不能缩进超过三个空白)。

\begin{lstlisting}
> This is a block quote. This
> paragraph has two lines.
>
> 1. This is a list inside a block quote.
> 2. Second item.
\end{lstlisting}

效果如下:

\begin{quote}
This is a block quote. This paragraph has two lines.

\begin{enumerate}
\def\labelenumi{\arabic{enumi}.}
\itemsep1pt\parskip0pt\parsep0pt
\item
  This is a list inside a block quote.
\item
  Second item.
\end{enumerate}
\end{quote}

有一个「偷懒」的形式:你只需要在引用区块的第一行行首输入\textgreater{}即可,后面的
行首可以省略符号:

\begin{lstlisting}
> This is a block quote. This
paragraph has two lines.

> 1. This is a list inside a block quote.
2. Second item.
\end{lstlisting}

由于区块引用可包含其他区块元素,而区块引用本身也是区块元素,所以,引用
是可以嵌套入其他引用的。

\begin{lstlisting}
> This is a block quote.
>
>> A block quote within a block quote.
\end{lstlisting}

\subsubsection{Extension:
blank\_before\_blockquote}\label{extension-blankux5fbeforeux5fblockquote}

原始markdown语法在区块引用之前并不需要预留空白行。Pandoc则需要(除非区
块引用位于文件最开始的地方)。这是因为以\textgreater{}符号开头的情况在一般文字段落中
相当常见(也许由于断行所致),这会导致非预期的格式。因此,除非是指定为
markdown\_strict格式,不然以下的语法在pandoc中将不会产生出嵌套区块引用:

\begin{lstlisting}
> This is a block quote.
>> Nested.
\end{lstlisting}

\section{代码}\label{ux4ee3ux7801}

\subsubsection{缩进式代码块}\label{ux7f29ux8fdbux5f0fux4ee3ux7801ux5757}

由四个空格或一个tab缩进的文本取做代码块,区块中的特殊字符、空格和换行都会
被保留,而缩进的空格和tab会在输出中移除,但在代码块中的空行不必缩进。

\begin{lstlisting}
#!/bin/bash

echo "Hello Markdown"
echo "Hello LaTeX"
\end{lstlisting}

\subsubsection{围栏式代码块}\label{ux56f4ux680fux5f0fux4ee3ux7801ux5757}

\textbf{Extension: fenced\_code\_blocks}

除了标准的缩进式代码块之外,Pandoc还支持围栏式代码块,
代码块以三个或三个以
上的\textasciitilde{}符号行开始,以等于或多于开始行\textasciitilde{}个数符号行结束,
若是代码块中含有\textasciitilde{},只需
使开始行和结束行中的\textasciitilde{}符号个数多于代码块中的即可

\begin{lstlisting}
~~~~~
~~~~
code here
~~~~
~~~~~~
\end{lstlisting}

\textbf{Extension: backtick\_code\_blocks}

与\lstinline!fenced_code_blocks!相同,只不过使用反引号 ` 替换波浪线
\textasciitilde{} 而已

\textbf{Extension: fenced\_code\_attributes}

\begin{lstlisting}[numbers=left, firstnumber=100, label=code:fencedcode, caption=code:fencedcode, float=htbp]
~~~~ {#code:mycode .haskell .numberLines startFrom="100"}
qsort []     = []
qsort (x:xs) = qsort (filter (< x) xs) ++ [x] ++
               qsort (filter (>= x) xs)
~~~~~~
\end{lstlisting}

这里的mycode为ID,haskell与numberLines是类别,而startsFrom则是值为
100的属性。numberLines和startFrom配合使用可以显示代码行号,如果没有
指定startFrom,则默认从1开始。有些输出格式可以利用这些信息来作语法
高亮。目前使用到这些信息的输出格式仅有HTML与LaTeX。如果指定的输出格
式及语言类别有语法高亮支持,那么上面那段代码区块将会以高亮并带有行号
的方式呈现。

仅指定高亮语言时,可以简写为以下形式:

\begin{lstlisting}
~~~haskell
qsort [] = []
~~~
\end{lstlisting}

\section{行区块}\label{ux884cux533aux5757}

\subsubsection{Extension: line\_blocks}\label{extension-lineux5fblocks}

行区块是一连串以竖线(
\textbar{})加上一个空格所构成的连续行。行与行间的区隔在
输出时将会以原样保留,行首的空白字元数目也一样会被保留;反之,这些行
将会以markdown的格式处理。这个语法在输入诗句或地址时很有帮助。

\begin{lstlisting}
| The limerick packs laughs anatomical
| In space that is quite economical.
|    But the good ones I've seen
|    So seldom are clean
| And the clean ones so seldom are comical

| 200 Main St.
| Berkeley, CA 94718
\end{lstlisting}

如果有需要的话,书写时也可以将完整一行拆成多行,但后续行必须以空白作为开始。
下面范例的前两行在输出时会被视为一整行:

\begin{lstlisting}
| The Right Honorable Most Venerable and Righteous Samuel L.
  Constable, Jr.
| 200 Main St.
| Berkeley, CA 94718
\end{lstlisting}

效果:

The limerick packs laughs anatomical\\In space that is quite
economical.\\\hspace*{0.333em}\hspace*{0.333em}\hspace*{0.333em}But the
good ones I've
seen\\\hspace*{0.333em}\hspace*{0.333em}\hspace*{0.333em}So seldom are
clean\\And the clean ones so seldom are comical

200 Main St.\\Berkeley, CA 94718

这是从reStructuredText借来的语法。

\section{列表}\label{ux5217ux8868}

\subsection{无序列表}\label{ux65e0ux5e8fux5217ux8868}

无序列表是以项目符号作列举的列表。每条项目都以项目符号( *, +或-)作开头。
下面是个简单的例子:

\begin{lstlisting}
* one
* two
* three
\end{lstlisting}

这会产生一个「紧凑」列表。如果你想要一个「宽松」列表,也就是说以段落格式处
理每个项目内的文字内容,那么只要在每个项目间加上空白行即可:

\begin{lstlisting}
* one

* two

* three
\end{lstlisting}

项目符号不能直接从行首最左边处输入,而必须以一至三个空白字元作缩进。项目符号
后必须跟着一个空白字元。

列表项目中的接续行,若与该项目的第一行文字对齐(在项目符号之后),看上去会较
为美观:

\begin{lstlisting}
* here is my first
  list item.
* and my second.
\end{lstlisting}

但markdown 也允许以下「偷懒」的格式:

\begin{lstlisting}
* here is my first
list item.
* and my second.
\end{lstlisting}

\subsubsection{四个空白规则}\label{ux56dbux4e2aux7a7aux767dux89c4ux5219}

一个列表项目可以包含多个段落以及其他区块等级的内容。然而,后续的段落必须接在空
白行之后,并且以四个空白或一个tab
作缩进。因此,如果项目里第一个段落与后面段落
对齐的话(也就是项目符号前置入两个空白),看上去会比较整齐美观:

\begin{lstlisting}
  * First paragraph.

    Continued.

  * Second paragraph. With a code block, which must be indented
    eight spaces:

        { code }
\end{lstlisting}

列表项目也可以包含其他列表。在这情况下前置的空白行是可有可无的。嵌套列表必须以四
个空白或一个tab 作缩进:

\begin{lstlisting}
* fruits
    + apples
        - macintosh
        - red delicious
    + pears
    + peaches
* vegetables
    + brocolli
    + chard
\end{lstlisting}

上一节提到,markdown
允许你以「偷懒」的方式书写,项目的接续行可以不和第一行对齐。不过,
如果一个列表项目中包含了多个段落或是其他区块元素,那么每个元素的第一行都必须缩进对齐。

\begin{lstlisting}
+ A lazy, lazy, list
item.

+ Another one; this looks
bad but is legal.

    Second paragraph of second
list item.
\end{lstlisting}

\textbf{注意:}尽管针对接续段落的「四个空白规则」是出自于官方的markdown
syntax guide,但是作
为对应参考用的Markdown.pl实作版本中并未遵循此一规则。所以当输入时若接续段落的缩进少于四
个空白时,pandoc所输出的结果会与Markdown.pl的输出有所出入。

在markdown syntax
guide中并未明确表示「四个空白规则」是否一体适用于所有位于列表项目里的
区块元素上;规范文件中只提及了段落与代码区块。但文件暗示了此规则适用于所有区块等级的内容
(包含嵌套列表),并且pandoc以此方向进行解读与实作。

\subsection{有序列表}\label{ux6709ux5e8fux5217ux8868}

有序列表与无序列表相类似,唯一的差别在于列表项目是以列举编号作开头,而不是项目符号。

在原始markdown
中,列举编号是阿拉伯数字后面接着一个句点与空白。数字本身代表的数值会被忽
略,因此下面两个列表并无差别:

\begin{lstlisting}
1.  one
2.  two
3.  three
\end{lstlisting}

上下两个列表的输出是相同的。

\begin{lstlisting}
5.  one
7.  two
1.  three
\end{lstlisting}

\subsubsection{Extension: fancy\_lists}\label{extension-fancyux5flists}

与原始markdown不同的是,Pandoc除了使用阿拉伯数字作为有序列表的编号外,也可以使用大写或
小写的英文字母,以及罗马数字。列表标记可以用括号包住,也可以单独一个右括号,抑或是句号。
如果列表标记是大写字母接着一个句号,句号后请使用至少两个空白字元。

\subsubsection{Extension: startnum}\label{extension-startnum}

除了列表标记外,Pandoc
也能判读列表的起始编号,这两项资讯都会保留于输出格式中。举例来说,
下面的输入可以产生一个从编号9
开始,以单括号为编号标记的列表,底下还跟着一个小写罗马数字 的子列表:

\begin{lstlisting}
 9)  Ninth
10)  Tenth
11)  Eleventh
       i. subone
      ii. subtwo
     iii. subthree
\end{lstlisting}

当遇到不同形式的列表标记时,Pandoc
会重新开始一个新的列表。所以,以下的输入会产生三份列表:

\begin{lstlisting}
(2) Two
(5) Three
1.  Four
*   Five
\end{lstlisting}

如果需要预设的有序列表标记符号,可以使用\#.:

\begin{lstlisting}
#.  one
#.  two
#.  three
\end{lstlisting}

\subsection{定义列表}\label{ux5b9aux4e49ux5217ux8868}

\subsubsection{Extension:
definition\_lists}\label{extension-definitionux5flists}

Pandoc支援定义列表,其语法的灵感来自于PHP Markdown
Extra以及reStructuredText:

\begin{lstlisting}
Term 1

:   Definition 1

Term 2 with *inline markup*

:   Definition 2

        { some code, part of Definition 2 }

    Third paragraph of definition 2.
\end{lstlisting}

每个专有名词(term)
都必须单独存在于一行,后面可以接着一个空白行,也可以省略,但一定
要接上一或多笔定义内容。一笔定义需由一个冒号或波浪线作开头,可以接上一或两个空白作为
缩进。定义本身的内容主体(包括接在冒号或波浪线后的第一行)应该以四个空白缩进。一个专
有名词可以有多个定义,而每个定义可以包含一或多个区块元素(段落、代码区块、列表等),
每个区块元素都要缩进四个空白或一个tab。

如果你在定义内容后面留下空白行(如同上面的范例),那么该段定义会被当作段落处理。在某
些输出格式中,这意谓著成对的专有名词与定义内容间会有较大的空白间距。在定义与定义之间,
以及定义与下个专有名词间不要留空白行,即可产生一个比较紧凑的定义列表:

\begin{lstlisting}
Term 1
  ~ Definition 1
Term 2
  ~ Definition 2a
  ~ Definition 2b
\end{lstlisting}

\subsection{编号范例列表}\label{ux7f16ux53f7ux8303ux4f8bux5217ux8868}

\subsubsection{Extension:
example\_lists}\label{extension-exampleux5flists}

这个特别的列表标记@可以用来产生连续编号的范例列表。列表中第一个以@标记的项目
会被编号为'1',接着编号为'2',依此类推,直到文件结束。范例项目的编号不会局限
于单一列表中,而是文件中所有以@为标记的项目均会次序递增其编号,直到最后一个。
举例如下:

\begin{lstlisting}
(@)  My first example will be numbered (1).
(@)  My second example will be numbered (2).

Explanation of examples.

(@)  My third example will be numbered (3).
\end{lstlisting}

编号范例可以加上标签,并且在文件的其他地方作参照:

\begin{lstlisting}
(@good)  This is a good example.

As (@good) illustrates, ...
\end{lstlisting}

标签可以是由任何英文字母、底线或是连字符号所组成的字串。

\subsection{紧凑与宽松列表}\label{ux7d27ux51d1ux4e0eux5bbdux677eux5217ux8868}

在与列表相关的「边界处理」上,Pandoc与Markdown.pl有着不同的处理结果。考虑如下代码:

\begin{lstlisting}
+   First
+   Second:
    -   Fee
    -   Fie
    -   Foe

+   Third
\end{lstlisting}

Pandoc会将以上列表转换为「紧凑列表」(在``First'',
``Second''或``Third''之中没有

标签), 而markdown则会在``Second''与``Third''
(但不包含``First'')里面置入\textless{}p\textgreater{}标签,这是因为``Third''
之前的空白行而造成的结果。Pandoc依循着一个简单规则:如果文字后面跟着空白行,那么就会被
视为段落。既然``Second''后面是跟着一个列表,而非空白行,那么就不会被视为段落了。至于子列
表的后面是不是跟着空白行,那就无关紧要了。(注意:即使是设定为markdown\_strict格式,
Pandoc仍是依以上方式处理列表项目是否为段落的判定。这个处理方式与markdown官方语法规范里
的描述一致,然而却与Markdown.pl的处理不同。)

\subsection{结束一个列表}\label{ux7ed3ux675fux4e00ux4e2aux5217ux8868}

如果你在列表之后放入一个缩排的代码区块,会有什么结果?

\begin{lstlisting}
-   item one
-   item two

    { my code block }
\end{lstlisting}

问题大了!这边pandoc(其他的markdown实作也是如此)会将\{ my code block
\}视为item two这个
列表项目的第二个段落来处理,而不会将其视为一个代码区块。

要在item
two之后「切断」列表,你可以插入一些没有缩排、输出时也不可见的内容,例如HTML的
注解:

\begin{lstlisting}
-   item one
-   item two

<!-- end of list -->

    { my code block }
\end{lstlisting}

当你想要两个各自独立的列表,而非一个大且连续的列表时,也可以运用同样的技巧:

\begin{lstlisting}
1.  one
2.  two
3.  three

<!-- -->

1.  uno
2.  dos
3.  tres
\end{lstlisting}

\section{分隔线}\label{ux5206ux9694ux7ebf}

一行中若包含三个以上的*,
-或\_符号(中间可以以空白字元分隔),则会产生一条分隔线:

\begin{lstlisting}
*  *  *  *

---------------
\end{lstlisting}

\begin{center}\rule{0.5\linewidth}{\linethickness}\end{center}

\section{表格}\label{ux8868ux683c}

有四种表格的形式可以使用。前三种适用于等宽字型的编辑环境,例如Courier。第四种则
不需要直行的对齐,因此可以在比例字型的环境下使用。

\subsection{简单表格}\label{ux7b80ux5355ux8868ux683c}

\subsubsection{Extension:
simple\_tables,table\_captions}\label{extension-simpleux5ftablestableux5fcaptions}

简单表格看起来像这样子:

\begin{lstlisting}
  Right     Left     Center     Default
-------     ------ ----------   -------
     12     12        12            12
    123     123       123          123
      1     1          1             1

Table:  Demonstration of simple table syntax.
\end{lstlisting}

\begin{longtable}[c]{@{}rlcl@{}}
\caption{Demonstration of simple table syntax.}\tabularnewline
\toprule
Right & Left & Center & Default\tabularnewline
\midrule
\endfirsthead
\toprule
Right & Left & Center & Default\tabularnewline
\midrule
\endhead
12 & 12 & 12 & 12\tabularnewline
123 & 123 & 123 & 123\tabularnewline
1 & 1 & 1 & 1\tabularnewline
\bottomrule
\end{longtable}

表头与资料列分别以一行为单位。直行的对齐则依照表头的文字和其底下虚线的相对位置来决定:

\begin{itemize}
\itemsep1pt\parskip0pt\parsep0pt
\item
  如果虚线与表头文字的右侧有切齐,而左侧比表头文字还长,则该直行为靠右对齐。
\item
  如果虚线与表头文字的左侧有切齐,而右侧比表头文字还长,则该直行为靠左对齐。
\item
  如果虚线的两侧都比表头文字长,则该直行为置中对齐。
\item
  如果虚线与表头文字的两侧都有切齐,则会套用预设的对齐方式(在大多数情况下,这将会是靠左对齐)。
\item
  表格底下必须接着一个空白行,或是一行虚线后再一个空白行。表格标题为可选的(上面的范例中有出现)。标题需是一个以Table:(或单纯只有:)开头作为前缀的段落,输出时前缀的这部份会被去除掉。表格标题可以放在表格之前或之后。
\end{itemize}

表头也可以省略,在省略表头的情况下,表格下方必须加上一行虚线以清楚标明表格的范围。例如:

\begin{lstlisting}
-------     ------ ----------   -------
     12     12        12             12
    123     123       123           123
      1     1          1              1
-------     ------ ----------   -------
\end{lstlisting}

当省略表头时,直行的对齐会以表格内容的第一行资料列决定。所以,以上面的表格为例,各直行的对齐依序会是靠右、靠左、置中以及靠右对齐。

多行表格 Extension: multiline\_tables,table\_captions

多行表格允许表头与表格资料格的文字能以复数行呈现(但不支援横跨多栏或纵跨多列的资料格)。以下为范例:

\begin{longtable}[c]{@{}clrl@{}}
\caption{Here's the caption. It, too, may span multiple lines.
看起来很像简单表格,但两者间有以下差别:}\tabularnewline
\toprule
\begin{minipage}[b]{0.15\columnwidth}\centering\strut
Centered Header
\strut\end{minipage} &
\begin{minipage}[b]{0.10\columnwidth}\raggedright\strut
Default Aligned
\strut\end{minipage} &
\begin{minipage}[b]{0.20\columnwidth}\raggedleft\strut
Right Aligned
\strut\end{minipage} &
\begin{minipage}[b]{0.31\columnwidth}\raggedright\strut
Left Aligned
\strut\end{minipage}\tabularnewline
\midrule
\endfirsthead
\toprule
\begin{minipage}[b]{0.15\columnwidth}\centering\strut
Centered Header
\strut\end{minipage} &
\begin{minipage}[b]{0.10\columnwidth}\raggedright\strut
Default Aligned
\strut\end{minipage} &
\begin{minipage}[b]{0.20\columnwidth}\raggedleft\strut
Right Aligned
\strut\end{minipage} &
\begin{minipage}[b]{0.31\columnwidth}\raggedright\strut
Left Aligned
\strut\end{minipage}\tabularnewline
\midrule
\endhead
\begin{minipage}[t]{0.15\columnwidth}\centering\strut
First
\strut\end{minipage} &
\begin{minipage}[t]{0.10\columnwidth}\raggedright\strut
row
\strut\end{minipage} &
\begin{minipage}[t]{0.20\columnwidth}\raggedleft\strut
12.0
\strut\end{minipage} &
\begin{minipage}[t]{0.31\columnwidth}\raggedright\strut
Example of a row that spans multiple lines.
\strut\end{minipage}\tabularnewline
\begin{minipage}[t]{0.15\columnwidth}\centering\strut
Second
\strut\end{minipage} &
\begin{minipage}[t]{0.10\columnwidth}\raggedright\strut
row
\strut\end{minipage} &
\begin{minipage}[t]{0.20\columnwidth}\raggedleft\strut
5.0
\strut\end{minipage} &
\begin{minipage}[t]{0.31\columnwidth}\raggedright\strut
Here's another one. Note the blank line between rows.
\strut\end{minipage}\tabularnewline
\bottomrule
\end{longtable}

在表头文字之前,必须以一列虚线作为开头(除非有省略表头)。
必须以一列虚线作为表格结尾,之后接一个空白行。
资料列与资料列之间以空白行隔开。
在多行表格中,表格分析器会计算各直行的栏宽,并在输出时尽可能维持各直行在原始文件中的相对比例。因此,要是你觉得某些栏位在输出时不够宽,你可以在markdown
的原始档中加宽一点。

和简单表格一样,表头在多行表格中也是可以省略的:

\begin{longtable}[c]{@{}clrl@{}}
\caption{Here's a multiline table without headers.
多行表格中可以单只包含一个资料列,但该资料列之后必须接着一个空白行(然后才是标示表格结尾的一行虚线)。如果没有此空白行,此表格将会被解读成简单表格。}\tabularnewline
\toprule
\endfirsthead
\toprule
\begin{minipage}[t]{0.15\columnwidth}\centering\strut
First
\strut\end{minipage} &
\begin{minipage}[t]{0.10\columnwidth}\raggedright\strut
row
\strut\end{minipage} &
\begin{minipage}[t]{0.20\columnwidth}\raggedleft\strut
12.0
\strut\end{minipage} &
\begin{minipage}[t]{0.31\columnwidth}\raggedright\strut
Example of a row that spans multiple lines.
\strut\end{minipage}\tabularnewline
\begin{minipage}[t]{0.15\columnwidth}\centering\strut
Second
\strut\end{minipage} &
\begin{minipage}[t]{0.10\columnwidth}\raggedright\strut
row
\strut\end{minipage} &
\begin{minipage}[t]{0.20\columnwidth}\raggedleft\strut
5.0
\strut\end{minipage} &
\begin{minipage}[t]{0.31\columnwidth}\raggedright\strut
Here's another one. Note the blank line between rows.
\strut\end{minipage}\tabularnewline
\bottomrule
\end{longtable}

格框表格 Extension: grid\_tables,table\_captions

格框表格看起来像这样:

: Sample grid table.

+---------------+---------------+--------------------+ \textbar{} Fruit
\textbar{} Price \textbar{} Advantages \textbar{}
+===============+===============+====================+ \textbar{}
Bananas \textbar{} \$1.34 \textbar{} - built-in wrapper \textbar{}
\textbar{} \textbar{} \textbar{} - bright color \textbar{}
+---------------+---------------+--------------------+ \textbar{}
Oranges \textbar{} \$2.10 \textbar{} - cures scurvy \textbar{}
\textbar{} \textbar{} \textbar{} - tasty \textbar{}
+---------------+---------------+--------------------+
以=串成的一行区分了表头与表格本体,这在没有表头的表格中也是可以省略的。在格框表格中的资料格可以包含任意的区块元素(复数段落、代码区块、清单等等)。不支援对齐,也不支援横跨多栏或纵跨多列的资料格。格框表格可以在Emacs
table mode下轻松建立。

管线表格 Extension: pipe\_tables,table\_captions

管线表格看起来像这样:

\begin{longtable}[c]{@{}rllc@{}}
\caption{Demonstration of simple table syntax. 这个语法与PHP markdown
extra中的表格语法相同。开始与结尾的管线字元是可选的,但各直行间则必须以管线区隔。上面范例中的冒号表明了对齐方式。表头可以省略,但表头下的水平虚线必须保留,因为虚线上定义了资料栏的对齐方式。}\tabularnewline
\toprule
Right & Left & Default & Center\tabularnewline
\midrule
\endfirsthead
\toprule
Right & Left & Default & Center\tabularnewline
\midrule
\endhead
12 & 12 & 12 & 12\tabularnewline
123 & 123 & 123 & 123\tabularnewline
1 & 1 & 1 & 1\tabularnewline
\bottomrule
\end{longtable}

因为管线界定了各栏之间的边界,表格的原始码并不需要像上面例子中各栏之间保持直行对齐。所以,底下一样是个完全合法(虽然丑陋)的管线表格:

fruit\textbar{} price -----\textbar{}-----: apple\textbar{}2.05
pear\textbar{}1.37 orange\textbar{}3.09
管线表格的资料格不能包含如段落、清单之类的区块元素,也不能包含复数行文字。

注意:Pandoc 也可以看得懂以下形式的管线表格,这是由Emacs 的orgtbl-mod
所绘制:

\begin{longtable}[c]{@{}ll@{}}
\toprule
One & Two\tabularnewline
\midrule
\endhead
my & table\tabularnewline
is & nice\tabularnewline
主要的差别在于以+取代了部分的 &
。其他的orgtbl功能并未支援。如果要指定非预设的直行对齐形式,你仍然需要在上面的表格中自行加入冒号。\tabularnewline
\bottomrule
\end{longtable}

\section{插图}\label{ux63d2ux56fe}

\section{表格}\label{ux8868ux683c-1}

Pandoc扩展的Markdown语法可以为代码块添加ID属性及语言类型属性,形如\lstinline!{#id .language}!,其中ID属性可以用来做交叉引用,使用\lstinline!\ref{id}!在正文中引用代码块。例如代码块\ref{code:demo}。

\backmatter

\end{document}
